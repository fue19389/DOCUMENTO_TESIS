En una sociedad donde la tecnología los métodos numéricos y cálculos computacionales superan las habilidades humanas, es necesario aplicar nuestro sentido de la empatía y humanidad para poder generar un verdadero desarrollo humano. Una de las ramas en la que se debe trabajar para lograr este desarrollo, es la inclusividad. Con este tema en mente, se plantea un proyecto que logre unir la visión de computadora y la robótica para generar un prototipo de agente móvil el cual pueda funcionar con base en gestos y movimientos faciales. Para esta tarea será importante lograr desarrollar sistemas efectivos de de software y hardware, así como la comunicación entre ellos. 

Para iniciar, se investigará y analizará la disponibilidad de tecnología, tanto software como hardware para implementar visión de computadora a nivel local. Esto se realizará por medio de cotizaciones y análisis de hojas de datos y manuales disponibles, así como la comparación entre plataformas de software. Luego, con los materiales definidos, se procederá a analizar y ajustar los algoritmos y plataformas a utilizar por separado. Después de verificado su correcto funcionamiento se empezará la etapa de desarrollo de algoritmos, en la que se combinarán y probarán las componentes de software, para ser verificadas únicamente por el sensor que se haya seleccionado previamente. Siguiendo esa línea de trabajo, el siguiente paso será verificar la integración del sistema de visión de computadora con simulaciones de agentes robóticos móviles. Cuando los resultados sean satisfactorios, se procederá a migrar este sistema a un agente móvil físico. 


