\section{Lenguaje de programación}
Inicialmente se inicia a trabajar utilizando herramientas presentadas en cursos de robótica, en este caso, se utiliza Python3.10 como lenguaje de programación. Dicha selección, resulta de la compatibilidad que tiene con \textit{OpenCV}, una librería utilizada para poder realizar aplicaciones de visión por computadora, que funciona tanto para leer las imágenes de entrenamiento a utilizar en el modelo de \textit{Machine Learning} como para la captura de imágenes en tiempo real. Es importante mencionar que, para realizar la programación, se optó por utilizar el IDE (Entorno de desarrollo integrado) \textit{PyCharm Community Edition}, dado que cuenta con comandos integrados para manejar entornos virtuales y conexiones con \textit{Git} y \textit{Github}. Además \textit{PyCharm} es uno de los IDE's que cuenta con una fácil incorporación al simulador \textit{Webots}, 

\section{Simulador}
En este caso se optó por utilizar un software de código abierto, de igual manera que la librería \textit{OpenCV}. Este simulador es conocido como \textit{Webots}, este simulador cuenta con diferentes opciones de ambientes y objetos para personalizar. Dicha personalización se refiere a las características espaciales y físicas, además de contar con diferentes agentes robóticos previamente incluidos. Dichos agentes, poseen el modelo virtual, así como la capacidad de controlarlos por medio de controladores en diferentes lenguajes de programación: C, C++, Java, Python, etc. 

\section{Sensor óptico}
\subsection{Cámara integrada en laptop}
Como primer recurso se tomó en cuenta el uso de la cámara incorporada en una laptop, dado que cualquier persona que esté trabajando en una, puede utilizar dicha herramienta, ahorrando así la compra de un módulo externo. Dicho sistema de cámara integrada puede variar su resolución desde los 0.4 (848$\times$480) Megapíxeles hasta los 2.1(1920$\times$1080). Este módulo se encuentra integrado en una laptop \textit{Lenovo IdeaPad Flex 5 16IAU7} y es fabricado por la empresa \textit{SunplusIT}, la cual es una empresa líder en la manufactura y distribución de chips para multimedia y aplicaciones automotrices, así como el proveedor de cámaras integradas para la empresa \textit{Lenovo}, por lo cual la replicabilidad de las secciones realizadas con dicho sensor es alta.

\subsection{Posible módulo externo}

\section{Transmisión de datos}
\section{Agente robótico}

%Esta etapa requiere verificar la disponibilidad de componentes a nivel local, tanto universitario y a nivel de país, para poder determinar el alcance de los prototipos. Luego de obtener el listado de opciones de sensores, se podrá hacer un análisis de la documentación para cada sensor y poder establecer el enfoque que se le dará, según las características del mismo. También se analizará la compatibilidad de cada sensor con diferentes lenguajes de programación y/o plataformas ya existentes. También, en esta etapa se realizará la búsqueda de robots disponibles para la implementación de los algoritmos propuestos. Por último, también se analizará cuales sensores externos (además de la cámara) se utilizarán para la parte de corrección de movimiento.

