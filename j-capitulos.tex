\chapter{Selección de hardware}

\section{Cámara integrada en laptop}

Como primer recurso se tomó en cuenta el uso de la cámara incorporada en una laptop, dado que cualquier persona que esté trabajando en una, puede utilizar dicha herramienta, ahorrando así la compra de un módulo externo. Dicho sistema de cámara integrada puede variar su resolución desde los 0.4 (848$\times$480) Megapíxeles hasta los 2.1(1920$\times$1080), teniendo
%Esta etapa requiere verificar la disponibilidad de componentes a nivel local, tanto universitario y a nivel de país, para poder determinar el alcance de los prototipos. Luego de obtener el listado de opciones de sensores, se podrá hacer un análisis de la documentación para cada sensor y poder establecer el enfoque que se le dará, según las características del mismo. También se analizará la compatibilidad de cada sensor con diferentes lenguajes de programación y/o plataformas ya existentes. También, en esta etapa se realizará la búsqueda de robots disponibles para la implementación de los algoritmos propuestos. Por último, también se analizará cuales sensores externos (además de la cámara) se utilizarán para la parte de corrección de movimiento.

\chapter{Ajuste de software y algoritmos}

\section{Software y bases de datos}
Inicialmente se inicia a trabajar utilizando herramientas presentadas en cursos de robótica, en este caso, se utiliza Python como lenguaje de programación. Dicha selección, resulta de la compatibilidad que tiene con \textit{OpenCV}, otro programa utilizado en cursos de robótica para poder aplicar visión por computadora


\chapter{Algoritmo de traducción}

Esta etapa contempla la iteración de pruebas para conseguir un algoritmo que funcione de manera efectiva al reconocer los gestos y movimientos de la cabeza del usuario. Luego de obtener un reconocimiento efectivo, se generarán diferentes ítems, clases o divisiones para cada tipo de gestos. Al tener definida la lista de tipos, se procede a utilizarlos para generar comandos genéricos para poder utilizar los agentes robóticos móviles simulados y físicos.

\chapter{Simulaciones}

El algoritmo anterior será utilizado para realizar simulaciones de movimiento/envío de comandos en diferentes escenarios. Para poder ejecutar dichas pruebas, será necesario también realizar un análisis de los diferentes simuladores disponibles, los cuales sean compatibles con los algoritmos determinados anteriormente. Se realizará un registro de datos de las iteraciones realizadas para analizarlas de manera estadísticas y determinar el comportamiento de cada algoritmo según el escenario propuesto. De esta manera, se podrá partir con datos o comandos iniciales para las pruebas físicas. Los escenarios a probar constan del entorno físico alrededor del robot, así como los diferentes gestos a utilizar. 

\chapter{Pruebas físicas}

Al obtener los parámetros iniciales determinados en las simulaciones, se empezará la migración de estos, junto con los algoritmos, hacia los robots a utilizar. En primer lugar se realizará una comparación entre los resultados obtenidos de manera física contra las simulaciones. Las pruebas físicas se realizarán en diferentes condiciones, como el de luminosidad, o el de contenido visual con diferentes niveles de ruido para tener un rango de operación más grande. También, las pruebas físicas tendrán un análisis cada algoritmo con diferentes variaciones en parámetros iniciales o en el algoritmo por sí mismo. De esta manera se podría verificar el alcance del sistema en conjunto, del robot, el sensor y el algoritmo seleccionado. También, se realizarán pruebas con diferentes sujetos de control, es decir diferentes personas realizando los gestos para controlar el movimiento del robot. Al realizar las pruebas, también se guardaran los datos obtenidos a partir de los sensores de tal manera que se puedan clasificar dependiendo del gesto utilizado. 

texto normalmente.