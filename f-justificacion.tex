Para llegar a ser una sociedad más inclusiva es necesario poder desarrollar tecnología que afronte diferentes situaciones, en este caso las personas con algún tipo de discapacidad. Existe una gran variedad de extremidades inhabilitadas en el espectro de las discapacidades. En muchos casos bastaría con un sistema de reconocimiento de gestos de manos para personas con parálisis de la cintura para abajo. Sin embargo, también hay que tomar en cuenta a las personas con discapacidades en las extremidades superiores, es ahí donde entra el reconocimiento de gestos de la cabeza. El desarrollo de un sistema de visión de computadora capaz de reconocer gestos faciales/de cabeza, que sea capaz de enviar comandos a un robot móvil es el primer paso para desarrollar una tecnología de apoyo. El control de un robot móvil por medio de este algoritmo, dará paso a una futura migración de comandos a plataformas móviles como sillas de ruedas, o bien vehículos eléctricos. Esta alternativa también busca utilizar sensores mucho más comunes en el mercado, como lo son los sensores de imágenes a comparación de sensores de profundidad o bien de señales bioeléctricas. Los sensores de imágenes, además tienen el valor agregado de brindar una mayor libertad de movimiento a los usuarios, ya que no son intrusivos, por lo que el gesto realizado tiende a una mayor naturalidad. Esto sucede dado que al utilizar aquellos que funcionan con EEG o EMG, requieren dispositivos de extensión al cuerpo humano, como son gorras, guantes o electrodos adheridos al cuerpo del sujeto de prueba. Por otro lado, también se espera que la sección de corrección por integración de sensores, permita una mayor seguridad al usuario. \cite{Wang_Jang_2018}, \cite{Bankar2015}





